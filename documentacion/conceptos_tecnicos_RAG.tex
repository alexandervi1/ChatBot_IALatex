\documentclass[12pt,a4paper]{article}

% --- Paquetes ---
\usepackage[utf8]{inputenc}
\usepackage[spanish]{babel}
\usepackage{geometry}
\usepackage{graphicx}
\usepackage{amsmath}
\usepackage{amssymb}
\usepackage{booktabs}
\usepackage{xcolor}
\usepackage{listings}
\usepackage{tikz}
\usepackage{hyperref}
\usepackage{fancyhdr}
\usepackage{tcolorbox}

% --- Configuración de página ---
\geometry{margin=2.5cm}
\pagestyle{fancy}
\fancyhf{}
\rhead{Chatbot IA con Copiloto LaTeX}
\lhead{Conceptos Técnicos}
\rfoot{Página \thepage}

% --- Colores personalizados ---
\definecolor{primaryblue}{RGB}{41, 128, 185}
\definecolor{secondarygreen}{RGB}{39, 174, 96}
\definecolor{accentorange}{RGB}{230, 126, 34}
\definecolor{codebg}{RGB}{245, 245, 245}

% --- Configuración de hyperref ---
\hypersetup{
    colorlinks=true,
    linkcolor=primaryblue,
    urlcolor=primaryblue,
    citecolor=secondarygreen
}

% --- Configuración de listings ---
\lstset{
    backgroundcolor=\color{codebg},
    basicstyle=\ttfamily\small,
    breaklines=true,
    frame=single,
    rulecolor=\color{gray},
    numbers=left,
    numberstyle=\tiny\color{gray}
}

% --- Cajas de información ---
\newtcolorbox{conceptbox}[1][]{
    colback=primaryblue!5,
    colframe=primaryblue,
    fonttitle=\bfseries,
    title=#1
}

\newtcolorbox{importantbox}{
    colback=accentorange!10,
    colframe=accentorange,
    fonttitle=\bfseries,
    title=Importante
}

% --- Título ---
\title{
    \vspace{-2cm}
    {\Huge \textcolor{primaryblue}{Conceptos Técnicos Fundamentales}}\\[0.5cm]
    {\Large Embeddings, Chunks, RAG y Búsqueda Híbrida}\\[0.3cm]
    {\large Chatbot IA con Copiloto LaTeX v4.2}
}
\author{Documentación Técnica para Exposición}
\date{\today}

\begin{document}

\maketitle
\tableofcontents
\newpage

% ============================================
\section{Chunk (Fragmento de Texto)}
% ============================================

\begin{conceptbox}[Definición]
Un \textbf{chunk} es un fragmento de texto de un documento largo, dividido en partes más pequeñas para poder procesarlo eficientemente.
\end{conceptbox}

\subsection{¿Por qué dividir en chunks?}

\begin{itemize}
    \item Los modelos de IA tienen \textbf{límites de texto} que pueden procesar
    \item Es más fácil buscar en fragmentos pequeños que en documentos enteros
    \item Permite encontrar la \textbf{parte específica} que responde una pregunta
\end{itemize}

\subsection{Ejemplo Visual}

\begin{figure}[h]
\centering
\begin{tikzpicture}[scale=0.8]
    % Documento original
    \draw[fill=gray!20, rounded corners] (0,0) rectangle (10,3);
    \node at (5,2.5) {\textbf{DOCUMENTO ORIGINAL (100 páginas)}};
    \node[align=center] at (5,1.2) {\small La inteligencia artificial es la simulación de\\
    \small procesos de inteligencia humana por parte de máquinas...};
    
    % Flecha
    \draw[->, thick, primaryblue] (5,0) -- (5,-0.8);
    \node[primaryblue] at (6.5,-0.4) {Chunking};
    
    % Chunks
    \draw[fill=secondarygreen!20, rounded corners] (0,-3.5) rectangle (3,-1.2);
    \node at (1.5,-1.5) {\footnotesize \textbf{CHUNK 1}};
    \node[align=center] at (1.5,-2.5) {\tiny "La inteligencia\\ \tiny artificial es..."};
    
    \draw[fill=secondarygreen!20, rounded corners] (3.5,-3.5) rectangle (6.5,-1.2);
    \node at (5,-1.5) {\footnotesize \textbf{CHUNK 2}};
    \node[align=center] at (5,-2.5) {\tiny "incluyen el\\ \tiny aprendizaje..."};
    
    \draw[fill=secondarygreen!20, rounded corners] (7,-3.5) rectangle (10,-1.2);
    \node at (8.5,-1.5) {\footnotesize \textbf{CHUNK 3}};
    \node[align=center] at (8.5,-2.5) {\tiny "reconocimiento\\ \tiny de voz..."};
\end{tikzpicture}
\caption{Proceso de división de un documento en chunks}
\end{figure}

\subsection{Configuración en la Aplicación}

\begin{itemize}
    \item Tamaño de cada chunk: $\approx$ \textbf{1000 caracteres}
    \item Se almacenan en la tabla \texttt{chunks} de PostgreSQL
    \item Cada chunk tiene su propio \textbf{embedding} asociado
\end{itemize}

\newpage
% ============================================
\section{Embedding (Vector Semántico)}
% ============================================

\begin{conceptbox}[Definición]
Un \textbf{embedding} es una representación numérica del significado de un texto. Es un vector de números que captura la ``esencia semántica'' de las palabras.
\end{conceptbox}

\subsection{¿Por qué convertir texto a números?}

\begin{itemize}
    \item Las computadoras \textbf{no entienden palabras}, solo números
    \item Los números permiten \textbf{medir similitud matemáticamente}
    \item Textos con significados parecidos $\rightarrow$ Vectores cercanos
\end{itemize}

\subsection{Representación Matemática}

Un embedding transforma texto en un vector de $n$ dimensiones:

\begin{equation}
\text{``perro''} \xrightarrow{\text{SentenceTransformer}} \vec{v} = [0.23, -0.45, 0.78, \ldots, 0.34] \in \mathbb{R}^{384}
\end{equation}

\subsection{Similitud entre Embeddings}

La similitud entre dos textos se calcula usando \textbf{similitud coseno}:

\begin{equation}
\text{sim}(\vec{a}, \vec{b}) = \frac{\vec{a} \cdot \vec{b}}{||\vec{a}|| \cdot ||\vec{b}||} = \frac{\sum_{i=1}^{n} a_i b_i}{\sqrt{\sum_{i=1}^{n} a_i^2} \cdot \sqrt{\sum_{i=1}^{n} b_i^2}}
\end{equation}

\begin{importantbox}
\begin{itemize}
    \item $\text{sim} \approx 1$ $\rightarrow$ Textos \textbf{muy similares}
    \item $\text{sim} \approx 0$ $\rightarrow$ Textos \textbf{sin relación}
    \item $\text{sim} \approx -1$ $\rightarrow$ Textos \textbf{opuestos}
\end{itemize}
\end{importantbox}

\subsection{Ejemplo de Similitudes}

\begin{table}[h]
\centering
\begin{tabular}{lcc}
\toprule
\textbf{Comparación} & \textbf{Similitud} & \textbf{Interpretación} \\
\midrule
``perro'' vs ``gato'' & 0.92 & Muy similares (animales) \\
``perro'' vs ``automóvil'' & 0.15 & Muy diferentes \\
``rey - hombre + mujer'' & $\approx$ ``reina'' & ¡Captura relaciones! \\
\bottomrule
\end{tabular}
\caption{Ejemplos de similitud coseno entre embeddings}
\end{table}

\subsection{Configuración en la Aplicación}

\begin{itemize}
    \item Modelo: \texttt{all-MiniLM-L6-v2} (Sentence Transformers)
    \item Dimensiones: \textbf{384 números} por embedding
    \item Almacenamiento: PostgreSQL con extensión \textbf{pgvector}
\end{itemize}

\newpage
% ============================================
\section{RAG (Retrieval-Augmented Generation)}
% ============================================

\begin{conceptbox}[Definición]
\textbf{RAG} es una técnica que combina \textbf{búsqueda} (Retrieval) con \textbf{generación de IA} (Generation). En vez de que la IA invente respuestas, primero BUSCA información relevante y luego genera una respuesta basada en esa información.
\end{conceptbox}

\subsection{Comparación: Con RAG vs Sin RAG}

\begin{table}[h]
\centering
\begin{tabular}{p{6cm}p{6cm}}
\toprule
\textbf{Sin RAG (ChatGPT normal)} & \textbf{Con RAG (Esta aplicación)} \\
\midrule
$\times$ Puede inventar datos falsos & $\checkmark$ Solo usa TUS documentos \\
$\times$ No conoce docs privados & $\checkmark$ Busca en tus PDFs \\
$\times$ No puede citar fuentes & $\checkmark$ Muestra origen de la info \\
$\times$ Conocimiento desactualizado & $\checkmark$ Conocimiento actualizado \\
\bottomrule
\end{tabular}
\caption{Ventajas de RAG sobre chatbots tradicionales}
\end{table}

\subsection{Pipeline RAG (4 Pasos)}

\begin{figure}[h]
\centering
\begin{tikzpicture}[scale=0.75, node distance=2.5cm]
    % Nodos
    \node[draw, rounded corners, fill=primaryblue!20, minimum width=2.5cm, minimum height=1cm] (pregunta) {Pregunta};
    \node[draw, rounded corners, fill=secondarygreen!20, minimum width=2.5cm, minimum height=1cm, right of=pregunta] (embed) {1. Embedding};
    \node[draw, rounded corners, fill=secondarygreen!20, minimum width=2.5cm, minimum height=1cm, right of=embed] (busqueda) {2. Búsqueda};
    \node[draw, rounded corners, fill=secondarygreen!20, minimum width=2.5cm, minimum height=1cm, right of=busqueda] (rerank) {3. Re-ranking};
    \node[draw, rounded corners, fill=accentorange!30, minimum width=2.5cm, minimum height=1cm, right of=rerank] (gen) {4. LLM};
    \node[draw, rounded corners, fill=primaryblue!20, minimum width=2.5cm, minimum height=1cm, right of=gen] (resp) {Respuesta};
    
    % Flechas
    \draw[->, thick] (pregunta) -- (embed);
    \draw[->, thick] (embed) -- (busqueda);
    \draw[->, thick] (busqueda) -- (rerank);
    \draw[->, thick] (rerank) -- (gen);
    \draw[->, thick] (gen) -- (resp);
\end{tikzpicture}
\caption{Pipeline RAG de 4 pasos}
\end{figure}

\subsection{Detalle de Cada Paso}

\begin{enumerate}
    \item \textbf{Embedding de la pregunta}: Convertir la pregunta del usuario a vector
    \item \textbf{Búsqueda en pgvector}: Encontrar los 50 chunks más similares
    \item \textbf{Re-ranking con CrossEncoder}: Reordenar por relevancia real $\rightarrow$ Top 5
    \item \textbf{Generación con LLM}: Enviar contexto + pregunta a Gemini/GPT/Claude
\end{enumerate}

\newpage
% ============================================
\section{Búsqueda Híbrida}
% ============================================

\begin{conceptbox}[Definición]
La \textbf{búsqueda híbrida} combina dos métodos de búsqueda complementarios: \textbf{semántica} (por significado) y \textbf{keywords} (por palabras exactas).
\end{conceptbox}

\subsection{Búsqueda Semántica (Vectorial)}

Busca por \textbf{SIGNIFICADO}, no por palabras exactas.

\begin{lstlisting}[language=SQL, caption=Query con pgvector]
SELECT content, 
       1 - (embedding <=> query_vector) AS similarity
FROM chunks
ORDER BY embedding <=> query_vector
LIMIT 50;
\end{lstlisting}

\textbf{Ejemplo:}
\begin{itemize}
    \item Pregunta: ``¿Cómo entrenar un modelo de ML?''
    \item Encuentra: ``El proceso de aprendizaje automático requiere...''
    \item $\rightarrow$ Aunque no dice ``entrenar'', el \textbf{significado} es similar
\end{itemize}

\subsection{Búsqueda por Keywords (BM25)}

Busca por \textbf{PALABRAS EXACTAS}.

\textbf{Ejemplo:}
\begin{itemize}
    \item Pregunta: ``¿Cómo entrenar un modelo de ML?''
    \item Encuentra: ``Para entrenar un modelo de machine learning...''
    \item $\rightarrow$ Coinciden las palabras ``entrenar'' y ``modelo''
\end{itemize}

\subsection{¿Por qué combinar ambas?}

\begin{table}[h]
\centering
\begin{tabular}{ll}
\toprule
\textbf{Método} & \textbf{Limitación} \\
\midrule
Semántica sola & Puede perder términos técnicos específicos \\
Keywords sola & No entiende sinónimos ni contexto \\
\textbf{Híbrida} & \textbf{¡Lo mejor de ambos mundos!} $\checkmark$ \\
\bottomrule
\end{tabular}
\end{table}

\subsection{Diagrama de Búsqueda Híbrida}

\begin{figure}[h]
\centering
\begin{tikzpicture}[scale=0.8]
    % Pregunta
    \node[draw, rounded corners, fill=primaryblue!20, minimum width=3cm] (q) at (5,5) {PREGUNTA};
    
    % Dos ramas de búsqueda
    \node[draw, rounded corners, fill=secondarygreen!20, minimum width=3cm] (sem) at (2,3) {Búsqueda Semántica};
    \node[draw, rounded corners, fill=secondarygreen!20, minimum width=3cm] (key) at (8,3) {Búsqueda Keywords};
    
    % Fusión
    \node[draw, rounded corners, fill=accentorange!20, minimum width=3cm] (fusion) at (5,1) {Fusión (RRF)};
    
    % Re-ranking
    \node[draw, rounded corners, fill=accentorange!20, minimum width=3cm] (rerank) at (5,-1) {Re-ranking};
    
    % Top 5
    \node[draw, rounded corners, fill=primaryblue!30, minimum width=3cm] (top) at (5,-3) {TOP 5 Chunks};
    
    % Flechas
    \draw[->, thick] (q) -- (sem);
    \draw[->, thick] (q) -- (key);
    \draw[->, thick] (sem) -- (fusion);
    \draw[->, thick] (key) -- (fusion);
    \draw[->, thick] (fusion) -- (rerank);
    \draw[->, thick] (rerank) -- (top);
    
    % Etiquetas
    \node[right] at (2.5,3) {\footnotesize Top 50};
    \node[left] at (7.5,3) {\footnotesize Top 50};
\end{tikzpicture}
\caption{Flujo de búsqueda híbrida con fusión y re-ranking}
\end{figure}

\newpage
% ============================================
\section{Resumen de Conceptos}
% ============================================

\begin{table}[h]
\centering
\begin{tabular}{p{2.5cm}p{5cm}p{5cm}}
\toprule
\textbf{Concepto} & \textbf{¿Qué es?} & \textbf{¿Para qué sirve?} \\
\midrule
\textbf{Chunk} & Fragmento de texto ($\approx$1000 chars) & Dividir documentos grandes \\
\textbf{Embedding} & Vector de 384 números & Representar significado numéricamente \\
\textbf{RAG} & Búsqueda + Generación & Responder con TUS documentos \\
\textbf{Búsqueda Híbrida} & Semántica + Keywords & Encontrar por significado Y palabras \\
\textbf{Re-ranking} & CrossEncoder & Ordenar por relevancia REAL \\
\bottomrule
\end{tabular}
\caption{Tabla resumen de todos los conceptos}
\end{table}

\section{Analogía Final: La Biblioteca Inteligente}

Imagina una \textbf{biblioteca con millones de libros}:

\begin{enumerate}
    \item \textbf{CHUNK} = Cada libro se divide en \textbf{páginas individuales}
    \item \textbf{EMBEDDING} = Cada página tiene una \textbf{etiqueta invisible} con su ``esencia''
    \item \textbf{BÚSQUEDA SEMÁNTICA} = Un bibliotecario que entiende \textbf{lo que QUIERES} aunque uses otras palabras
    \item \textbf{BÚSQUEDA KEYWORDS} = Un índice tradicional que busca \textbf{palabras exactas}
    \item \textbf{RE-RANKING} = Un experto que revisa los resultados y selecciona \textbf{los más útiles}
    \item \textbf{RAG} = El bibliotecario te \textbf{resume} la información de las mejores páginas
\end{enumerate}

\vfill
\begin{center}
\textcolor{gray}{\small Documento generado para exposición académica --- Chatbot IA con Copiloto LaTeX v4.2}
\end{center}

\end{document}
