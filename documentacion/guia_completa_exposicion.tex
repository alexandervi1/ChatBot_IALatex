\documentclass[12pt,a4paper]{article}

% --- Paquetes ---
\usepackage[utf8]{inputenc}
\usepackage[spanish]{babel}
\usepackage{geometry}
\usepackage{graphicx}
\usepackage{amsmath}
\usepackage{booktabs}
\usepackage{xcolor}
\usepackage{listings}
\usepackage{tikz}
\usepackage{hyperref}
\usepackage{fancyhdr}
\usepackage{tcolorbox}
\usepackage{fontawesome5}
\usepackage{enumitem}

\usetikzlibrary{shapes.geometric, arrows, positioning, fit}

% --- Configuración de página ---
\geometry{margin=2.5cm}
\pagestyle{fancy}
\fancyhf{}
\rhead{Chatbot IA con Copiloto LaTeX v4.2}
\lhead{Guía Completa para Exposición}
\rfoot{Página \thepage}

% --- Colores ---
\definecolor{frontend}{RGB}{59, 130, 246}
\definecolor{backend}{RGB}{34, 197, 94}
\definecolor{database}{RGB}{249, 115, 22}
\definecolor{cache}{RGB}{168, 85, 247}
\definecolor{ai}{RGB}{236, 72, 153}
\definecolor{codebg}{RGB}{245, 245, 245}

% --- Hyperref ---
\hypersetup{
    colorlinks=true,
    linkcolor=frontend,
    urlcolor=frontend
}

% --- Listings ---
\lstset{
    backgroundcolor=\color{codebg},
    basicstyle=\ttfamily\small,
    breaklines=true,
    frame=single,
    rulecolor=\color{gray}
}

% --- Cajas ---
\newtcolorbox{frontendbox}{
    colback=frontend!10,
    colframe=frontend,
    fonttitle=\bfseries,
    title={\faReact\ Frontend}
}

\newtcolorbox{backendbox}{
    colback=backend!10,
    colframe=backend,
    fonttitle=\bfseries,
    title={\faPython\ Backend}
}

\newtcolorbox{databasebox}{
    colback=database!10,
    colframe=database,
    fonttitle=\bfseries,
    title={\faDatabase\ Base de Datos}
}

\newtcolorbox{aibox}{
    colback=ai!10,
    colframe=ai,
    fonttitle=\bfseries,
    title={\faRobot\ Inteligencia Artificial}
}

\newtcolorbox{conceptbox}[1][]{
    colback=gray!5,
    colframe=gray!50,
    fonttitle=\bfseries,
    title=#1
}

% --- Título ---
\title{
    \vspace{-2cm}
    {\Huge \textcolor{frontend}{Guía Completa para Exposición}}\\[0.5cm]
    {\Large Chatbot IA con Copiloto LaTeX}\\[0.3cm]
    {\normalsize Versión 4.2 --- Explicación Sencilla de Toda la Arquitectura}
}
\author{Documentación Técnica}
\date{\today}

\begin{document}

\maketitle
\tableofcontents
\newpage

% ============================================
\section{¿Qué es esta Aplicación?}
% ============================================

\begin{conceptbox}[Resumen Ejecutivo]
Es un \textbf{chatbot inteligente} que permite:
\begin{enumerate}
    \item \textbf{Subir documentos} (PDF, Word, TXT)
    \item \textbf{Hacer preguntas} sobre esos documentos
    \item \textbf{Recibir respuestas} basadas en el contenido real
    \item \textbf{Escribir documentos LaTeX} con ayuda de IA
\end{enumerate}
\end{conceptbox}

\subsection{¿Para quién es?}
\begin{itemize}
    \item \faGraduationCap\ \textbf{Estudiantes}: Consultar apuntes y libros con IA
    \item \faBook\ \textbf{Investigadores}: Analizar papers académicos
    \item \faEdit\ \textbf{Académicos}: Escribir artículos LaTeX con asistencia
\end{itemize}

\subsection{Tecnología Principal: RAG}

\textbf{RAG} = Retrieval-Augmented Generation (Generación Aumentada por Recuperación)

\begin{center}
\fbox{En vez de que la IA \textbf{invente}, primero \textbf{BUSCA} en tus documentos}
\end{center}

% ============================================
\section{Arquitectura General}
% ============================================

\begin{figure}[h]
\centering
\begin{tikzpicture}[
    node distance=1.5cm,
    box/.style={draw, rounded corners, minimum width=3.5cm, minimum height=1cm, align=center},
    arrow/.style={->, thick}
]
    % Usuario
    \node[box, fill=gray!20] (user) {\faUser\ Usuario};
    
    % Frontend
    \node[box, fill=frontend!20, below=of user] (front) {\faReact\ Frontend\\(Next.js)};
    
    % Backend
    \node[box, fill=backend!20, below=of front] (back) {\faPython\ Backend\\(FastAPI)};
    
    % Servicios
    \node[box, fill=database!20, below left=1.5cm and 0.5cm of back] (db) {\faDatabase\ PostgreSQL\\+ pgvector};
    \node[box, fill=purple!20, below=of back] (redis) {\faMemory\ Redis\\(Cache)};
    \node[box, fill=ai!20, below right=1.5cm and 0.5cm of back] (ai) {\faRobot\ IA\\(Gemini/GPT)};
    
    % Flechas
    \draw[arrow] (user) -- node[right]{\footnotesize HTTP} (front);
    \draw[arrow] (front) -- node[right]{\footnotesize API REST} (back);
    \draw[arrow] (back) -- (db);
    \draw[arrow] (back) -- (redis);
    \draw[arrow] (back) -- (ai);
\end{tikzpicture}
\caption{Arquitectura general de la aplicación}
\end{figure}

\subsection{Los 5 Servicios (Docker)}

\begin{table}[h]
\centering
\begin{tabular}{llll}
\toprule
\textbf{Servicio} & \textbf{Puerto} & \textbf{Tecnología} & \textbf{Función} \\
\midrule
Frontend & 3000 & Next.js 15 & Interfaz de usuario \\
Backend & 8000 & FastAPI & API y lógica \\
Base de Datos & 5432 & PostgreSQL & Almacenamiento \\
Cache & 6379 & Redis & Velocidad \\
Worker & -- & Celery & Tareas pesadas \\
\bottomrule
\end{tabular}
\caption{Servicios de la aplicación}
\end{table}

\newpage
% ============================================
\section{Frontend (Lo que ve el usuario)}
% ============================================

\begin{frontendbox}
El frontend es la \textbf{interfaz gráfica} que el usuario ve en su navegador.
\end{frontendbox}

\subsection{Tecnologías Usadas}

\begin{table}[h]
\centering
\begin{tabular}{lp{8cm}}
\toprule
\textbf{Tecnología} & \textbf{¿Qué hace?} \\
\midrule
\textbf{Next.js 15} & Framework de React para páginas web rápidas \\
\textbf{React 19} & Librería para crear interfaces interactivas \\
\textbf{TypeScript} & JavaScript con tipos (menos errores) \\
\textbf{Tailwind CSS} & Estilos CSS rápidos de escribir \\
\textbf{Radix UI} & Componentes accesibles (botones, menús, etc.) \\
\textbf{CodeMirror} & Editor de código para LaTeX \\
\bottomrule
\end{tabular}
\end{table}

\subsection{Páginas Principales}

\begin{enumerate}
    \item \textbf{/} --- Página de inicio y login
    \item \textbf{/chat} --- Chat con documentos
    \item \textbf{/copilot} --- Editor LaTeX con IA
    \item \textbf{/admin} --- Panel de administración
\end{enumerate}

\subsection{¿Cómo se comunica con el Backend?}

\begin{lstlisting}[language=JavaScript, caption=Ejemplo de llamada al API]
// Enviar mensaje al chat
const response = await fetch('http://localhost:8000/chat/message', {
    method: 'POST',
    headers: {
        'Authorization': 'Bearer <token>',
        'Content-Type': 'application/json'
    },
    body: JSON.stringify({ message: "Que es IA?" })
});
\end{lstlisting}

\subsection{Componentes Importantes}

\begin{itemize}
    \item \texttt{chat/} --- Interfaz del chat (mensajes, input, fuentes)
    \item \texttt{copilot/} --- Editor LaTeX (33 archivos)
    \item \texttt{admin/} --- Panel de administración
    \item \texttt{ui/} --- Componentes reutilizables (botones, dialogs)
\end{itemize}

\newpage
% ============================================
\section{Backend (La lógica del servidor)}
% ============================================

\begin{backendbox}
El backend es el \textbf{cerebro} de la aplicación. Procesa las peticiones, busca en la base de datos y genera respuestas.
\end{backendbox}

\subsection{Tecnologías Usadas}

\begin{table}[h]
\centering
\begin{tabular}{lp{8cm}}
\toprule
\textbf{Tecnología} & \textbf{¿Qué hace?} \\
\midrule
\textbf{FastAPI} & Framework web ultrarrápido para Python \\
\textbf{Python 3.11} & Lenguaje de programación principal \\
\textbf{SQLAlchemy} & ORM para comunicarse con la base de datos \\
\textbf{Pydantic} & Validación automática de datos \\
\textbf{Celery} & Ejecutar tareas pesadas en segundo plano \\
\textbf{JWT} & Tokens de autenticación seguros \\
\bottomrule
\end{tabular}
\end{table}

\subsection{Routers (Endpoints del API)}

\begin{table}[h]
\centering
\begin{tabular}{llp{6cm}}
\toprule
\textbf{Router} & \textbf{Ruta} & \textbf{Función} \\
\midrule
\texttt{auth} & /auth/* & Login, registro, JWT \\
\texttt{chat} & /chat/* & Enviar mensajes, RAG \\
\texttt{documents} & /documents/* & Subir/eliminar archivos \\
\texttt{providers} & /providers/* & Configurar proveedor IA \\
\texttt{admin} & /admin/* & Gestión de usuarios \\
\bottomrule
\end{tabular}
\end{table}

\subsection{Servicios (La lógica de negocio)}

\begin{table}[h]
\centering
\begin{tabular}{lp{7cm}}
\toprule
\textbf{Servicio} & \textbf{¿Qué hace?} \\
\midrule
\texttt{search\_engine.py} & Búsqueda híbrida + generación de respuestas \\
\texttt{ai\_providers.py} & Conectar con Gemini, GPT, Claude, Ollama \\
\texttt{embedding\_system.py} & Convertir texto a vectores \\
\texttt{auth\_service.py} & Login, tokens, seguridad \\
\texttt{pdf\_processor.py} & Extraer texto de PDFs \\
\bottomrule
\end{tabular}
\end{table}

\subsection{¿Cómo funciona una petición?}

\begin{figure}[h]
\centering
\begin{tikzpicture}[node distance=2cm, auto]
    \node[draw, rounded corners, fill=frontend!20] (1) {1. Usuario envía mensaje};
    \node[draw, rounded corners, fill=backend!20, right=of 1] (2) {2. FastAPI recibe};
    \node[draw, rounded corners, fill=database!20, right=of 2] (3) {3. Busca en BD};
    \node[draw, rounded corners, fill=ai!20, below=of 3] (4) {4. Envía a IA};
    \node[draw, rounded corners, fill=frontend!20, left=of 4] (5) {5. Devuelve respuesta};
    
    \draw[->, thick] (1) -- (2);
    \draw[->, thick] (2) -- (3);
    \draw[->, thick] (3) -- (4);
    \draw[->, thick] (4) -- (5);
\end{tikzpicture}
\caption{Flujo de una petición}
\end{figure}

\newpage
% ============================================
\section{Base de Datos (Dónde se guarda todo)}
% ============================================

\begin{databasebox}
PostgreSQL es la base de datos relacional. La extensión \textbf{pgvector} permite guardar y buscar \textbf{vectores} (embeddings).
\end{databasebox}

\subsection{Tablas Principales}

\begin{table}[h]
\centering
\begin{tabular}{lp{8cm}}
\toprule
\textbf{Tabla} & \textbf{¿Qué guarda?} \\
\midrule
\textbf{users} & Usuarios (email, contraseña hasheada, rol) \\
\textbf{documents} & Documentos subidos (nombre, dueño, fecha) \\
\textbf{chunks} & Fragmentos de texto + \textbf{embedding} (vector 384) \\
\textbf{chats} & Conversaciones de cada usuario \\
\textbf{messages} & Mensajes individuales del chat \\
\textbf{audit\_logs} & Registro de acciones de seguridad \\
\bottomrule
\end{tabular}
\end{table}

\subsection{¿Qué es pgvector?}

Es una \textbf{extensión de PostgreSQL} que permite:

\begin{itemize}
    \item Guardar vectores de números (embeddings)
    \item Buscar vectores similares (similitud coseno)
    \item Crear índices para búsquedas rápidas
\end{itemize}

\begin{lstlisting}[language=SQL, caption=Búsqueda vectorial con pgvector]
-- Buscar los 5 chunks mas similares a mi pregunta
SELECT content, 
       1 - (embedding <=> query_vector) AS similitud
FROM chunks
ORDER BY embedding <=> query_vector
LIMIT 5;
\end{lstlisting}

El operador \texttt{<=>} calcula la \textbf{distancia coseno} entre vectores.

\subsection{Redis (Cache)}

\begin{itemize}
    \item Guarda embeddings ya calculados (no recalcular)
    \item Rate limiting (límite de peticiones por minuto)
    \item TTL de 7 días para embeddings
\end{itemize}

\newpage
% ============================================
\section{Inteligencia Artificial}
% ============================================

\begin{aibox}
La IA es el componente que \textbf{entiende} las preguntas y \textbf{genera} las respuestas usando el contexto encontrado.
\end{aibox}

\subsection{Proveedores de IA Soportados}

\begin{table}[h]
\centering
\begin{tabular}{llll}
\toprule
\textbf{Proveedor} & \textbf{Modelos} & \textbf{API Key} & \textbf{Costo} \\
\midrule
Google Gemini & gemini-2.5-flash/pro & Sí & Gratis con límites \\
OpenAI & gpt-4o, gpt-4o-mini & Sí & Pago por uso \\
Anthropic & claude-3.5-sonnet & Sí & Pago por uso \\
Cerebras & llama-3.3-70b & Sí & Gratis con límites \\
\textbf{Ollama} & qwen2.5:3b, llama3.2 & \textbf{No} & \textbf{100\% Gratis} \\
\bottomrule
\end{tabular}
\end{table}

\subsection{Sentence Transformers (Embeddings)}

\begin{itemize}
    \item Modelo: \texttt{all-MiniLM-L6-v2}
    \item Función: Convertir texto $\rightarrow$ vector de 384 números
    \item Uso: Búsqueda semántica (por significado)
\end{itemize}

\subsection{CrossEncoder (Re-ranking)}

\begin{itemize}
    \item Modelo: \texttt{cross-encoder/ms-marco-MiniLM-L-6-v2}
    \item Función: Reordenar resultados por relevancia REAL
    \item Uso: De Top-50 candidatos $\rightarrow$ Top-5 mejores
\end{itemize}

\subsection{Pipeline RAG Completo}

\begin{figure}[h]
\centering
\begin{tikzpicture}[node distance=1.8cm, auto, 
    box/.style={draw, rounded corners, minimum width=2.5cm, minimum height=0.8cm, align=center, font=\small}]
    
    \node[box, fill=gray!20] (q) {Pregunta};
    \node[box, fill=ai!20, right=of q] (emb) {Embedding};
    \node[box, fill=database!20, right=of emb] (search) {Búsqueda\\Híbrida};
    \node[box, fill=purple!20, right=of search] (rerank) {Re-ranking};
    \node[box, fill=ai!20, below=1cm of rerank] (llm) {LLM\\(Gemini)};
    \node[box, fill=backend!20, left=of llm] (resp) {Respuesta};
    
    \draw[->, thick] (q) -- (emb);
    \draw[->, thick] (emb) -- (search);
    \draw[->, thick] (search) -- node[above]{\tiny Top 50} (rerank);
    \draw[->, thick] (rerank) -- node[right]{\tiny Top 5} (llm);
    \draw[->, thick] (llm) -- (resp);
\end{tikzpicture}
\caption{Pipeline RAG de la aplicación}
\end{figure}

\newpage
% ============================================
\section{Seguridad}
% ============================================

\subsection{Autenticación (JWT)}

\begin{itemize}
    \item \textbf{Access Token}: Expira en 30 minutos
    \item \textbf{Refresh Token}: Expira en 7 días, se rota en cada uso
    \item \textbf{Algoritmo}: HS256 (HMAC con SHA-256)
\end{itemize}

\begin{lstlisting}[caption=Estructura de un JWT]
Header: {"alg": "HS256", "typ": "JWT"}
Payload: {"sub": "user_id", "exp": 1234567890}
Signature: HMACSHA256(header + payload, SECRET_KEY)
\end{lstlisting}

\subsection{Encriptación}

\begin{table}[h]
\centering
\begin{tabular}{lll}
\toprule
\textbf{Dato} & \textbf{Método} & \textbf{Reversible} \\
\midrule
Contraseñas & BCrypt (hash) & No \\
API Keys & Fernet (AES-128) & Sí \\
\bottomrule
\end{tabular}
\end{table}

\subsection{Rate Limiting}

\begin{table}[h]
\centering
\begin{tabular}{ll}
\toprule
\textbf{Rol} & \textbf{Límite} \\
\midrule
Anónimo & 30 peticiones/minuto \\
Usuario & 100 peticiones/minuto \\
Admin & 1000 peticiones/minuto \\
\bottomrule
\end{tabular}
\end{table}

% ============================================
\section{Docker y Despliegue}
% ============================================

\subsection{¿Qué es Docker?}

Docker es una herramienta que \textbf{empaqueta} toda la aplicación (código + dependencias) en \textbf{contenedores} que funcionan igual en cualquier computadora.

\subsection{docker-compose.yml}

Orquesta \textbf{5 contenedores} que trabajan juntos:

\begin{lstlisting}[caption=Servicios en docker-compose.yml]
services:
  frontend:   # Next.js en puerto 3000
  backend:    # FastAPI en puerto 8000
  db:         # PostgreSQL en puerto 5432
  cache:      # Redis en puerto 6379
  celery:     # Worker para tareas async
\end{lstlisting}

\subsection{Comando para Levantar Todo}

\begin{lstlisting}[language=bash]
# Construir e iniciar todos los servicios
docker-compose up --build -d

# Ver logs
docker-compose logs -f

# Parar todo
docker-compose down
\end{lstlisting}

\newpage
% ============================================
\section{Resumen: Preguntas de Exposición}
% ============================================

\begin{table}[h]
\centering
\begin{tabular}{p{5cm}p{8cm}}
\toprule
\textbf{Pregunta} & \textbf{Respuesta Corta} \\
\midrule
¿Qué tecnología usa el Frontend? & Next.js 15 con React 19 y TypeScript \\
\midrule
¿Qué tecnología usa el Backend? & FastAPI con Python 3.11 \\
\midrule
¿Qué base de datos usa? & PostgreSQL 16 con extensión pgvector \\
\midrule
¿Qué es RAG? & Buscar en documentos antes de generar respuesta \\
\midrule
¿Qué es un embedding? & Representación numérica del significado del texto \\
\midrule
¿Qué es un chunk? & Fragmento de ~1000 caracteres de un documento \\
\midrule
¿Cómo se autentican los usuarios? & JWT (Access + Refresh Token) \\
\midrule
¿Cómo se guardan las contraseñas? & Hash con BCrypt (irreversible) \\
\midrule
¿Para qué sirve Redis? & Cache de embeddings y rate limiting \\
\midrule
¿Para qué sirve Celery? & Procesar tareas pesadas en segundo plano \\
\midrule
¿Qué proveedores de IA soporta? & Gemini, OpenAI, Claude, Cerebras, Ollama \\
\midrule
¿Puede funcionar offline? & Sí, con Ollama (modelos locales) \\
\bottomrule
\end{tabular}
\caption{Preguntas frecuentes para la exposición}
\end{table}

\section{Diagrama Final de Arquitectura}

\begin{figure}[h]
\centering
\begin{tikzpicture}[scale=0.9, transform shape,
    layer/.style={draw, rounded corners, minimum width=12cm, minimum height=1.5cm, align=center}]
    
    % Capas
    \node[layer, fill=frontend!20] (ui) at (0,6) {\textbf{FRONTEND} --- Next.js 15 + React 19 + TypeScript + Tailwind};
    
    \node[layer, fill=backend!20] (api) at (0,4) {\textbf{BACKEND} --- FastAPI + Python 3.11 + SQLAlchemy + JWT};
    
    \node[layer, fill=ai!20] (ai) at (0,2) {\textbf{IA} --- Sentence Transformers + CrossEncoder + LLMs (Gemini/GPT)};
    
    \node[layer, fill=database!20] (data) at (0,0) {\textbf{DATOS} --- PostgreSQL + pgvector + Redis + Celery};
    
    % Flechas
    \draw[->, thick] (ui) -- (api);
    \draw[->, thick] (api) -- (ai);
    \draw[->, thick] (api) -- (data);
    \draw[->, thick] (ai) -- (data);
    
\end{tikzpicture}
\caption{Arquitectura en capas de la aplicación}
\end{figure}

\vfill
\begin{center}
\textcolor{gray}{\rule{0.8\textwidth}{0.5pt}}\\[0.5cm]
{\Large \textbf{¡Buena suerte en tu exposición!}}\\[0.3cm]
\textcolor{gray}{Chatbot IA con Copiloto LaTeX v4.2}
\end{center}

\end{document}
